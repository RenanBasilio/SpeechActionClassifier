\title{Diarização de Locutor em Conteúdo de Vídeo Baseada em Análise de Expressão Facial via Aprendizado de Máquina Supervisionado}
\foreigntitle{Facial Expression Analysis for Speaker Diarization and Identification in Video Content via Supervised Machine Learning}
\author{Renan Fasolato}{Basilio}
\advisor{Prof.}{Geraldo Zimbrão}{da Silva}{D.Sc.}

\examiner{Prof.}{Geraldo Zimbrão da Silva}{D.Sc.}
\examiner{Prof.}{Jano Moreira de Souza}{Ph.D.}
\examiner{Prof.}{Heraldo Luís Silveira de Almeida}{D.Sc.}

\department{ECI}

% Confira a tabela a seguir para saber como preencher o comando \department de acordo com seu curso (Graduação - Poli,Graduação - EQ) ou programa (Pós-Graduação - COPPE).


%%%%%% Para alunos da EQ %%%%%%

%% Course											Option
%% Engenharia Química                               EQ
%% Engenharia de Bioprocessos                       EB
%% Engenharia de Alimentos                          EAL
%% Química industrial                               QI

%%%%%% Para alunos da POLI %%%%%%

%% Course											Option
%% Engenharia Ambiental                             EA
%% Engenharia Civil                                 ECV
%% Engenharia de Computação e Informação            ECI
%% Engenharia de Controle e Automação               ECA
%% Engenharia de Materiais                          EMAT
%% Engenharia de Petróleo                           EPT
%% Engenharia de Produção                           EPR
%% Engenharia Eletrônica e de Computação            EEC
%% Engenharia Elétrica                              EET
%% Engenharia Mecânica                              EMC
%% Engenharia Metalúrgica                           EMET
%% Engenharia Naval e Oceânica                      ENO
%% Engenharia Nuclear                               ENU


%%%%%% Para alunos da COPPE %%%%%%

%% Program											Option
%% Engenharia Biomédica								PEB
%% Engenharia Civil									PEC
%% Engenharia Elétrica								PEE
%% Engenharia Mecânica								PEM
%% Engenharia Metalúrgica e de Materiais				PEMM
%% Engenharia Nuclear									PEN
%% Engenharia Oceânica								PENO
%% Planejamento Energético							PPE
%% Engenharia de Produção								PEP
%% Engenharia Química									PEQ
%% Engenharia de Sistemas e Computação				PESC
%% Engenharia de Transportes							PET

\date{\the\month}{\the\year}

\keyword{Aprendizado Supervisionado}
\keyword{Aprendizado de Máquina}
\keyword{Diarização de Locutor}

\maketitle
\frontmatter

\makecatalog

\dedication{Aos meus pais, que sempre me apoiaram, e minha noiva Allison, que esteve sempre ao meu lado e esperou por mim durante toda a minha jornada universitária.}

\begin{abstract}
Este trabalho apresenta uma prova de conceito para um sistema diarizador baseado em uma rede neural convolucional capaz de identificar o estado de fala de um locutor a partir de um vídeo do mesmo, sem fazer uso da onda de áudio relacionada, para aplicação em casos onde esta se encontre em baixa qualidade, ruidosa, ou mesmo ausente. 
Para isso, é realizado um pré-processamento sobre a imagem de entrada de forma a identificar a posição da face do locutor e extrair desta suas feições principais, que servem de entrada para a rede neural. 
Uma arquitetura para a rede neural baseada em uma VGG, modificada para lidar com dados tridimensionais, foi construída, cuja implementação levou a um modelo com acurácia preditiva de 86.56\%, resultando em uma taxa de erro de diarização de 32.5 sobre os dados de teste no melhor caso.
\end{abstract}

\begin{foreignabstract}
This work introduces a proof of concept for a speaker diarization system based on a convolutional neural network capable of identifying speech in a frontal video of a speaker without making use of the associatied audio wave, for use cases in which the latter is either low quality, noisy, or outright missing.
For this purpose we extract facial landmarks from each frame of the input video, and feed them into the neural network.
We also propose an architecture for the prediction model based on a VGG deep neural network modified to handle three-dimensional data, with which we obtained a predictive accuracy of 86.56\% on the test dataset, which resulted in a diarization error rate of 32.5.
\end{foreignabstract}

\tableofcontents

\listoffigures

\listoftables