\chapter{Fundamentação Teórica}

\section{Redes Neurais Artificiais}
\label{sec:dnn}

Redes Neurais Artificiais são um conjunto de algoritmos inspirados pelo funcionamento do cérebro humano, introduzidos pela primeira vez em 1943 quando Warren McCulloch e Walter Pitts modelaram o funcionamento de neurônios através de circuitos elétricos simples\cite{mccullochLogicalCalculusIdeas1943}. 
Esse modelo continuou evoluindo ao longo dos anos, culminando no desenvolvimento do \textit{Perceptron} por Rosenblatt em 1958 \cite{rosenblattPerceptronProbabilisticModel1958} (figura \ref{fig:perceptron}), um algoritmo de classificação binária semelhante a uma rede neural composta por um único neurônio.

No entanto, devido à falta de capacidade computacional e ao grande volume de dados necessários para o treinamento de redes neurais, estes algoritmos permaneceram apenas um conceito acadêmico durante vários anos.
Somente em 2006, Geoffrey Hinton proporia formalmente o conceito de rede neural profunda \cite{hintonFastLearningAlgorithm2006}, um tipo de rede neural composto por múltiplas camadas de neurônios.

\begin{figure}[ht]
    \centering
    \begin{tikzpicture}[]
    \node[coordinate] (input) [] {};
    \node[state, circle, align=center] (input1) at (input) [] {$x_2$};
    \node[state, circle, align=center] (input2) [above=of input1, yshift=-0.3cm] {$x_1$};
    \node[state, circle, align=center] (input3) [below=of input1, yshift=0.3cm] {$x_3$};
    \node[draw=none, fill=none] [below=of input3, yshift=-0.3cm] (inputs_label) {Entrada};

    \node[coordinate] (hidden1) [right=of input, xshift=1.5cm] {};
    \node[state, circle, align=center] (hidden11) at (hidden1) [] {};
        \draw [->, color=gray] (input1) -- (hidden11);
        \draw [->, color=gray] (input2) -- (hidden11);
        \draw [->, color=gray] (input3) -- (hidden11);
    \node[state, circle, align=center] (hidden12) [above=of hidden11, yshift=-0.7cm] {};
        \draw [->, color=gray] (input1) -- (hidden12);
        \draw [->, color=gray] (input2) -- (hidden12);
        \draw [->, color=gray] (input3) -- (hidden12);
    \node[state, circle, align=center] (hidden13) [above=of hidden12, yshift=-0.7cm] {};
        \draw [->, color=gray] (input1) -- (hidden13);
        \draw [->, color=gray] (input2) -- (hidden13);
        \draw [->, color=gray] (input3) -- (hidden13);
    \node[state, circle, align=center] (hidden14) [below=of hidden11, yshift=0.7cm] {};
        \draw [->, color=gray] (input1) -- (hidden14);
        \draw [->, color=gray] (input2) -- (hidden14);
        \draw [->, color=gray] (input3) -- (hidden14);
    \node[state, circle, align=center] (hidden15) [below=of hidden14, yshift=0.7cm] {};
        \draw [->, color=gray] (input1) -- (hidden15);
        \draw [->, color=gray] (input2) -- (hidden15);
        \draw [->, color=gray] (input3) -- (hidden15);

    \node[coordinate] (hidden2) [right=of hidden1, xshift=1.5cm] {};
    \node[state, circle, align=center] (hidden21) [above=of hidden2, yshift=-0.7cm] {};
        \draw [->, color=gray] (hidden11) -- (hidden21);
        \draw [->, color=gray] (hidden12) -- (hidden21);
        \draw [->, color=gray] (hidden13) -- (hidden21);
        \draw [->, color=gray] (hidden14) -- (hidden21);
        \draw [->, color=gray] (hidden15) -- (hidden21);
    \node[state, circle, align=center] (hidden22) [above=of hidden21, yshift=-0.7cm] {};
        \draw [->, color=gray] (hidden11) -- (hidden22);
        \draw [->, color=gray] (hidden12) -- (hidden22);
        \draw [->, color=gray] (hidden13) -- (hidden22);
        \draw [->, color=gray] (hidden14) -- (hidden22);
        \draw [->, color=gray] (hidden15) -- (hidden22);
    \node[state, circle, align=center] (hidden23) [below=of hidden2, yshift=0.7cm] {};
        \draw [->, color=gray] (hidden11) -- (hidden23);
        \draw [->, color=gray] (hidden12) -- (hidden23);
        \draw [->, color=gray] (hidden13) -- (hidden23);
        \draw [->, color=gray] (hidden14) -- (hidden23);
        \draw [->, color=gray] (hidden15) -- (hidden23);
    \node[state, circle, align=center] (hidden24) [below=of hidden23, yshift=0.7cm] {};
        \draw [->, color=gray] (hidden11) -- (hidden24);
        \draw [->, color=gray] (hidden12) -- (hidden24);
        \draw [->, color=gray] (hidden13) -- (hidden24);
        \draw [->, color=gray] (hidden14) -- (hidden24);
        \draw [->, color=gray] (hidden15) -- (hidden24);
    \node[draw=none, fill=none] [right=of inputs_label, xshift=0.25cm] (hidden_label) {Camadas Ocultas};

    \node[coordinate] (output) [right=of hidden2, xshift=1.5cm] {};
    \node[state, circle, align=center] (output1) [above=of output, yshift=-0.75cm] {$y_1$};
        \draw [->, color=gray] (hidden21) -- (output1);
        \draw [->, color=gray] (hidden22) -- (output1);
        \draw [->, color=gray] (hidden23) -- (output1);
        \draw [->, color=gray] (hidden24) -- (output1);
    \node[state, circle, align=center] (output2) [below=of output, yshift=0.75cm] {$y_2$};
        \draw [->, color=gray] (hidden21) -- (output2);
        \draw [->, color=gray] (hidden22) -- (output2);
        \draw [->, color=gray] (hidden23) -- (output2);
        \draw [->, color=gray] (hidden24) -- (output2);
    \node[draw=none, fill=none] [right=of hidden_label, xshift=0.25cm] (output_label) {Saída};

\end{tikzpicture}
    \caption{Ilustração de uma rede neural profunda com 3 camadas.}
    \label{fig:dnn}
\end{figure}

Recentemente, com o desenvolvimento de algoritmos para o treinamento de redes neurais em GPU (do Inglês \textit{Graphics Processing Unit}), que possibilitaram o treinamento mais rápido e eficiente de redes neurais de múltiplas camadas, assim como o crescente volume de dados disponível para treinamento, o tópico de redes neurais vem rapidamente ganhando popularidade.
Algoritmos desse tipo já são utilizados em diversas áreas da ciência, incluindo aplicações em medicina \cite{hannunCardiologistLevelArrhythmiaDetection2019,phamPredictingHealthcareTrajectories2017,houDeepSFDeepConvolutional2018}, previsão do tempo \cite{akramSequenceSequenceWeather2016}, veículos autônomos \cite{bojarskiEndEndLearning2016,yudinObjectDetectionDeep2019}, entre outras \cite{abiodunStateoftheartArtificialNeural2018}.

Tratam-se de algoritmos próprios para problemas que envolvem o reconhecimento de padrões, e a capacidade de treina-los a partir de dados existentes os torna eficazes para aplicação em tópicos nos quais as relações entre os dados de entrada não sejam totalmente conhecidas.

\subsection{Neurônios Artificiais}

Biologicamente, quando o potencial elétrico na base do axônio de um neurônio chega a um valor pré-determinado, por meio do acúmulo de sinais de entrada recebidos pelos dendritos, um pulso elétrico é gerado e propagado até as sinapses, terminais nos quais o neurônio conecta com os demais.
Trata-se de um sinal binário; o neurônio pode ou não estar ativado a cada instante. No entanto, a frequência relativa das ativações pode conter informação \cite{behnkeHierarchicalNeuralNetworks2003}.

\begin{figure}[H]
    \centering
    \includesvg[width=0.7\textwidth]{neurono-ido.svg}
    \caption{A estrutura de um neurônio biológico. Imagem adaptada de \cite{dhp1080IdoSkemoPri2016}.}
    \label{fig:bio_neuron}
\end{figure}

Funções de Ativação, também conhecidas como funções de propagação ou neurônios artificiais quando aplicadas no contexto de redes neurais, buscam modelar de forma simplificada esses aspectos biológicos.
Elas recebem como entrada os valores resultantes das camadas anteriores, ponderados por pesos específicos a cada conexão, e, quando estes valores atingem um limiar de excitação pré-determinado, os transformam em uma única saída.

\begin{figure}[H]
    \centering
    \begin{tikzpicture}
    \node[state, circle, align=center] (activation) [] {$\phi(x)$};
    
    \node[state, circle, align=center] (sum) [left=of activation] {$\sum$};
        \draw [->] (sum) -- (activation);
        
    \node[state, circle, align=center] (input1) [above left=of sum, yshift=-1cm, xshift=-1cm] {$x_2$};
        \draw [->] (input1) -- node [midway, above] {$w_2$} (sum);
        
    \node[state, circle, align=center] (input0) [above=of input1, yshift=-0.5cm] {$x_1$};
        \draw [->] (input0) -- node [midway, above] {$w_1$} (sum);
        
    \node[state, circle, align=center] (input2) [below left=of sum, yshift=1cm, xshift=-1cm] {$x_3$};
        \draw [->] (input2) -- node [midway, above] {$w_3$} (sum);
        
    \node[state, circle, align=center] (input3) [below=of input2, yshift=-0.5cm] {$x_n$};
        \path (input2) -- (input3) node [font=\Huge, midway, sloped] {$\dots$};
        \draw [->] (input3) -- node [midway, above=0.5em] {$w_n$} (sum);
        
    \node[draw=none, fill=none] [right=of activation] (output) {Saída};
        \draw [->] (activation) -- (output);
        
    \node[draw=none, fill=none] [below=of input3, yshift=0.7cm] (inputs) {Entradas};

\end{tikzpicture}
    \caption{O \textit{Perceptron}, uma rede neural com um único neurônio.}
    \label{fig:perceptron}
\end{figure}

Geralmente, a função de ativação é constante e independente do processo de treinamento da rede neural; o ajuste é feito sobre os pesos das conexões, através algoritmo de retropropagação \cite{dreyfusArtificialNeuralNetworks1990}.
O algoritmo é capaz de, a partir do resultado de uma camada, calcular o efeito de cada peso sobre o gradiente da função.
Assim, ele possibilita a utilização de algoritmos tradicionais de otimização para encontrar os pesos locais de cada conexão, mesmo em redes com múltiplas camadas de centenas de neurônios, em função do efeito global destes.

Comumente, são utilizadas como função de ativação as funções ReLU (\textit{Rectified Linear Unit}), tangente hiperbólica, sigmoide, e Heaviside, em função do intervalo de saída e comportamento desejado.

\begin{equation}\label{eq:relu}
    \phi(x) = \left\{
        \begin{array}{ll}
            0, & x \le 0\\
            x, & x > 0
        \end{array}
    \right.
\end{equation}

A função ReLU (equação \ref{eq:relu}), utilizada neste trabalho, tem comportamento tal que sua ativação ocorre sempre que $x>0$, com o valor resultante sendo o próprio valor de $x$. 
Essa função é ideal para aplicações em redes neurais profundas pois o gradiente é preservado após cada camada da rede, o que não ocorre com as funções sigmoide e tangente hiperbólica.
Nessas, o gradiente tende a valores infinitesimalmente pequenos após camadas sucessivas, dificultando o uso do algoritmo de retropropagação.
Além disso, o treinamento de redes neurais profundas que utilizam a função ReLU é mais rápido do que o das outras funções de ativação, devido à sua simplicidade \cite{krizhevskyImageNetClassificationDeep2017}.

\subsection{Redes Neurais Convolucionais}
\label{sec:convnet}

Redes Neurais Convolucionais são redes neurais compostas por uma ou mais camadas de convolução.

\section{Reconhecimento Facial}
\label{sec:facialrecog}

\section{Detecção de Marcadores Faciais}
\label{sec:faciallm}