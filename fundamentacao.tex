\chapter{Fundamentação Teórica}

\section{Reconhecimento Facial}
\label{sec:facialrecog}

\section{Detecção de Marcadores Faciais}
\label{sec:faciallm}

\section{Redes Neurais Profundas}
\label{sec:dnn}

Redes Neurais Artificiais são um conjunto de algoritmos inspirados pelo funcionamento do cérebro humano, introduzidos pela primeira vez no ano de 1943 quando Warren McCulloch e Walter Pitts modelaram o funcionamento de neurônios através de circuitos elétricos\cite{mccullochLogicalCalculusIdeas1943}.
Esse modelo continuou evoluindo ao longo dos anos, culminando no desenvolvimento do \textit{Perceptron} por Rosenblatt em 1958\cite{rosenblattPerceptronProbabilisticModel1958}, um tipo de algoritmo de classificação binária baseado em uma rede de neurônios artificiais com uma única camada densamente conectada.
No entanto, devido à capacidade computacional e ao volume de dados necessários para o treinamento de redes neurais, essas permaneceram apenas um conceito acadêmico durante vários anos; somente no ano de 2006, Geoffrey Hinton proporia pela primeira vez o conceito de Redes Neurais Profundas\cite{hintonFastLearningAlgorithm2006}, um tipo de rede neural composto por múltiplas camadas de neurônios.

% Figure, Deep Neural Networks



\subsection{Redes Neurais Convolucionais}
\label{sec:convnet}