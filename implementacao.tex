\chapter{Implementação}
\label{chap:impl}

Neste capítulo discutiremos tópicos relacionados à implementação do sistema de diarização de locutor proposto. 

Na seção \ref{sec:tools} apresentamos as ferramentas principais utilizadas no desenvolvimento deste trabalho. Em seguida, na seção \ref{sec:preproc}, apresentamos o trabalho realizado para preparação e pré-processamento dos dados. Na seção \ref{sec:sysarch} discutimos a arquitetura do sistema desenvolvido, discussão esta que aprofundamos na seção \ref{sec:topology} através da demonstração da topologia da rede neural utilizada.

\section{Ferramentas}
\label{sec:tools}

Nesta seção apresentamos as principais ferramentas utilizadas durante a implementação deste trabalho. Definiremos suas principais características, assim como as funcionalidades das mesmas que foram utilizadas, e as motivações por trás de sua escolha.

Na seção \ref{subsec:dlib} apresentaremos a biblioteca dlib, utilizada para propósito de reconhecimento e demarcação facial no projeto. Em seguida, na seção \ref{subsec:tf} apresentaremos a biblioteca Tensorflow, utilizada por sua robusta implementação de rede neural. Na seção \ref{subsec:environ} apresentaremos o ambiente utilizado para treinamento do modelo, assim como seus recursos computacionais. Por fim, na seção seção \ref{subsec:otools}, acreditaremos brevemente as demais ferramentas utilizadas em caráter pontual no trabalho, e que, portanto não receberão seções dedicadas.

\subsection{dlib}
\label{subsec:dlib}
\cite{dlib09}
% TODO: Escrever sobre a dlib

\subsection{Tensorflow}
\label{subsec:tf}
\cite{tensorflow2015-whitepaper}
% TODO: Escrever sobre Tensorflow

\subsection{Ambiente de Desenvolvimento}
\label{subsec:environ}



\subsection{Outras Ferramentas}
\label{subsec:otools}

Nesta seção apresentamos as demais bibliotecas utilizadas no desenvolvimento do projeto. Em cada subseção descrevemos brevemente a biblioteca, definindo seu papel no projeto.

As bibliotecas se encontram ordenadas por sua função na pipeline do classificador, discutida de forma mais aprofundada na seção \ref{sec:sysarch}.

\subsubsection{Jupyter Notebook}

O Jupyter Notebook \cite{Kluyver:2016aa} foi a IDE utilizada na maior parte do desenvolvimento do classificador, por permitir rápida visualização e alteração de seções individuais do código. Essa capacidade foi fundamental para validação de várias funcionalidades implementadas, tais como o carregamento de vídeo, a produção da imagem intermediária com os marcadores faciais, e o cálculo do fluxo ótico denso.

\subsubsection{OpenCV}

OpenCV \cite{opencv_library} é uma biblioteca de código aberto para aplicações de Visão Computacional. Ela foi utilizada para realizar a leitura quadro a quadro dos arquivos de vídeo a serem processados pelo sistema, e para codificar em video a saída do classificador.

\subsubsection{Matplotlib}

A Matplotlib \cite{Hunter:2007} é uma biblioteca para produção de gráficos e imagens em Python. Ela foi utilizada para produzir as imagens intermediárias, através do desenho de polígonos a partir dos vértices produzidos pelo processo de marcação facial.

\subsubsection{Pandas}

Pandas \cite{mckinney-proc-scipy-2010} é uma biblioteca de processamento de dados em Python. Ela foi utilizada no préprocessamento dos dados com a finalidade de manipular os arquivos csv produzidos pelo processo de diarização manual dos videos do dataset de depoimentos.

\section{Preparação dos Dados}
\label{sec:preproc}

% TODO: Escrever sobre o préprocessamento dos dados

\section{Arquitetura do Sistema}
\label{sec:sysarch}

% TODO: Escrever sobre a arquitetura do sistema

\section{Topologia da Rede Neural}
\label{sec:topology}

% TODO: Escrever sobre a topologia da rede neural