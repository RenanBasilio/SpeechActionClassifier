\chapter{Resultados}
\label{chap:results}

Neste capítulo discutiremos os resultados obtidos nos testes realizados com o sistema diarizador.
Primeiramente, na seção \ref{sec:results-model}, discutiremos os resultados referentes ao modelo classificador treinado como componente preditora do sistema.
Depois, na seção \ref{sec:results-app}, apresentaremos os resultados referentes à aplicação diarizadora como um todo.

\section{Modelo Classificador}
\label{sec:results-model}

Nesta seção avaliaremos o desempenho do modelo treinado em relação à sua capacidade de classificar corretamente segmentos de vídeo de 15 quadros.
Para essa finalidade, utilizamos os segmentos de 2 vídeos distintos do dataset que não foram utilizados anteriormente durante o treinamento do modelo, totalizando 990 segmentos válidos com duração de meio segundo e classificados manualmente.
Além disso, utilizamos também versões destes espelhadas horizontalmente, totalizando 1980 amostras totais para teste do modelo.

Na seção \ref{sec:results-model-gen} apresentamos as métricas gerais de desempenho do modelo, e na seção \ref{sec:results-model-confusion} apresentamos as métricas de confusão, referentes ao desempenho deste em relação a cada uma das classes alvo.
Todas as métricas serão apresentadas para 3 conjuntos de pesos distintos obtidos durante o processo treinamento do modelo. Estes são:
\begin{itemize}
    \item Os pesos que minimizam o valor da custo utilizada;
    \item Os pesos que maximizam o valor da acurácia do modelo;
    \item Os pesos obtidos ao final de todas as épocas do treinamento.
\end{itemize}

\subsection{Métricas Gerais}
\label{sec:results-model-gen}

As métricas gerais calculadas para o modelo incluem a sua acurácia, que representa a capacidade do modelo de classificar um dado segmento de video corretamente, a entropia categórica (equação \ref{eq:categorical_crossentropy}), que corresponde à certeza do modelo quanto a suas predições, e a área abaixo da curva ROC.
A figura \ref{fig:general-metrics-val} mostra a evolução dessas métricas ao longo do processo de treinamento do modelo, enquanto a tabela \ref{tab:general-metrics-val} destaca os valores destas métricas para os conjuntos de pesos obtidos ao final do treinamento.

\begin{figure}[ht]
    \centering
    \begin{tikzpicture}
        \begin{axis}[xmin=0, ymin=0, xmax=68, ymax=1, width=0.65\textwidth, every axis plot/.append style={very thick}, minor tick num=5, grid=both, grid style={line width=.1pt, draw=gray!10}, legend style={at={(1.18,0.02)},anchor=south east}, no markers, legend cell align={left}]
            \addplot table [x=Step, y=Value, col sep=comma] {figures/run-2020-05-21-2357_validation-tag-epoch_basics_accuracy.csv};
            \addlegendentry{Acurácia}
            \addplot[OliveGreen] table [x=Step, y=Value, col sep=comma] {figures/run-2020-05-21-2357_validation-tag-epoch_basics_auc.csv};
            \addlegendentry{Area Under Curve}
            \addplot[red] table [x=Step, y=Value, col sep=comma] {figures/run-2020-05-21-2357_validation-tag-epoch_basics_cce.csv};
            \addlegendentry{Entropia Categórica}
        \end{axis}
    \end{tikzpicture}
    \caption{Evolução das métricas gerais sobre o conjunto de validação\\ durante o treinamento do modelo. A linha azul representa a evolução da acurácia, enquanto a linha verde corresponde à evolução da área abaixo da curva ROC, e a linha vermelha à da entropia categórica cruzada.}
    \label{fig:general-metrics-val}
\end{figure}

\begin{table}[ht]
    \centering
    \begin{tabular}{|c|c|c|c|c|}
        \rule{1cm}{0pt}&\rule{1.5cm}{0pt}&\rule{1.5cm}{0pt}&\rule{1.5cm}{0pt}&\rule{1cm}{0pt}\\[-\arraystretch\normalbaselineskip]
        \hline
        Pesos & Época & Acurácia & AUC & Entropia Categórica \\
        \hline
        \hline
        Custo Mínimo & 18 & 0.8596 & 0.9314 & 0.3382 \\
        \hline
        Acurácia Máxima & 39 & 0.8656 & 0.9336 & 0.3429 \\
        \hline
        Final & 68 & 0.8561 & 0.9236 & 0.4198 \\
        \hline
    \end{tabular}
    \caption{Métricas referentes aos conjuntos de pesos obtidos ao final\\ do processo de treinamento.}
    \label{tab:general-metrics-val}
\end{table}

\subsection{Métricas de Confusão}
\label{sec:results-model-confusion}

As métricas de confusão demonstram a capacidade do modelo de classificar corretamente os itens pertencentes a cada uma das classes.
Essas incluem a matriz de confusão, que visualiza as classificações feitas pelo diarizador em relação à classe real de cada item, revelando as tendências deste, e os valores de precisão, \textit{recall}, e \textit{$F_1$ Score} calculados a partir desta.

\begin{figure}[ht]
    \centering
    \resizebox{0.9\textwidth}{!}{
        \begin{tabular}{cc}
            \MakeConfusionMatrix{855}{93}{185}{847} & \MakeConfusionMatrix{845}{103}{163}{869} \\
            Menor Custo & Maior Acurácia \\
            & \\
            \multicolumn{2}{c}{\MakeConfusionMatrix{872}{76}{209}{823}} \\
            \multicolumn{2}{c}{Final}
        \end{tabular}
    }
    \caption{Matrizes de Confusão para os conjuntos de pesos obtidos ao final \\ do treinamento do modelo. }
    \label{fig:confusion_matrices}
\end{figure}

A figura \ref{fig:confusion_matrices} mostra as matrizes de confusão dos conjuntos de pesos treinados que obtiveram melhor desempenho sobre o conjunto de treinamento.
Observa-se a acurácia do modelo no percentual de valores que se encontram da diagonal principal desta; Esses valores correspondem às classificações corretas dos elementos de cada classe.
Note, ainda, que os conjuntos de pesos de menor custo e finais do treinamento tendem à esquerda, indicando uma tendência do modelo a classificar negativamente os segmentos observados.

Além disso, para melhor entendimento das propriedades do modelo treinado, calculamos a precisão e o \textit{recall} correspondente a cada classe, assim como o \textit{$F_1$ Score} para os conjuntos de pesos obtidos ao final do processo de treinamento do modelo.
A precisão, calculada a partir da equação \ref{eq:precision}, representa a probabilidade de que um elemento classificado como pertencente a uma classe pertença realmente à mesma, enquanto a taxa de \textit{recall}, calculada a partir da equação \ref{eq:recall}, representa a capacidade do modelo de classificar corretamente os elementos pertencentes a uma determinada classe.
Por fim, o \textit{$F_1$ Score}, calculado através da equação \ref{eq:f1-score}, corresponde à média harmônica dessas duas métricas, representando a acurácia do modelo na classificação de uma determinada classe.

\begin{figure}[ht]
    \centering
    \resizebox{0.9\textwidth}{!}{
        \begin{tabular}{cc}
            \begin{tikzpicture}
                    \begin{axis}[xmin=0, ymin=0, xmax=68, ymax=1, every axis plot/.append style={very thick}, minor tick num=5, grid=both, grid style={line width=.1pt, draw=gray!10}, legend style={at={(0.98,0.02)},anchor=south east}, no markers]
                        \addplot table [x=Step, y=Value, col sep=comma] {figures/run-2020-05-21-2357_validation-tag-epoch_confusion_precision_idle.csv};
                        \addplot table [x=Step, y=Value, col sep=comma] {figures/run-2020-05-21-2357_validation-tag-epoch_confusion_precision_speak.csv};
                    \end{axis}
                \end{tikzpicture} & \begin{tikzpicture}
                    \begin{axis}[xmin=0, ymin=0, xmax=68, ymax=1, every axis plot/.append style={very thick}, minor tick num=5, grid=both, grid style={line width=.1pt, draw=gray!10}, legend style={at={(0.98,0.02)},anchor=south east}, no markers, legend cell align={left}]
                        \addplot table [x=Step, y=Value, col sep=comma] {figures/run-2020-05-21-2357_validation-tag-epoch_confusion_recall_idle.csv};
                        \addlegendentry{Idle}
                        \addplot[red] table [x=Step, y=Value, col sep=comma] {figures/run-2020-05-21-2357_validation-tag-epoch_confusion_recall_speak.csv};
                        \addlegendentry{Speech}
                    \end{axis}
                \end{tikzpicture} \\
            Precisão & \textit{Recall} \\
        \end{tabular}
    }
    \caption{Evolução da Precisão e \textit{Recall} de cada classe sobre o conjunto de validação durante o processo de treinamento do modelo. A linha azul representa a classe \textit{Idle}, que corresponde à ausência de fala no fragmento, enquanto a linha vermelha representa a classe \textit{Speech}, que corresponde à detecção de fala no fragmento.}
    \label{fig:class_precision_recall}
\end{figure}

\begin{equation}\label{eq:precision}
    Precis\Tilde{a}o = \frac{TP}{TP + FP}
\end{equation}

\begin{equation}\label{eq:recall}
    Recall = \frac{TP}{TP + FN}
\end{equation}

\begin{equation}\label{eq:f1-score}
    F_1 = 2 \times \frac{precis\Tilde{a}o \times recall}{precis\Tilde{a}o + recall}
\end{equation}

\begin{table}[ht]
    \centering
    \begin{tabular}{|c|c|c|c|c|c|c|}
        \hline
         & \multicolumn{3}{c|}{Idle} & \multicolumn{3}{c|}{Speech}  \\
        \hline
        Pesos & Precisão & \textit{Recall} & $F_1$ & Precisão & \textit{Recall} & $F_1$ \\
        \hline
        \hline
        Custo Mínimo & 0.8221 & 0.9019 & 0.8601 & 0.9011 & 0.8207 & 0.8590 \\
        \hline
        Acurácia Máxima & 0.8383 & 0.8940 & 0.8652 & 0.8914 & 0.8421 & 0.8660 \\
        \hline
        Final & 0.8067 & 0.9155 & 0.8577 & 0.9198 & 0.7975 & 0.8543 \\
        \hline
    \end{tabular}
    \caption{Métricas de confusão para os conjuntos de pesos obtidos ao final de processo de treinamento do modelo.}
    \label{tab:confusion_metrics_val}
\end{table}

A figura \ref{fig:class_precision_recall} demonstra a evolução das métricas precisão e \textit{recall} calculadas a partir da matriz de confusão ao longo do processo de treinamento para cada classe, enquanto a tabela \ref{tab:confusion_metrics_val} explicita o seu valor para os conjuntos de pesos obtidos ao final do treinamento.

\section{Aplicação Diarizadora}
\label{sec:results-app}

Nesta seção avaliaremos o desempenho da aplicação diarizadora.
Para isso, utilizaremos como métricas de comparação a Taxa de Erro de Diarização (DER, do inglês \textit{Diarization Error Rate}) e a Taxa de Erro de Jaccard (JER, do inglês \textit{Jaccard Error Rate}).

\begin{equation}\label{eq:der}
    DER = \frac{False\ Alarm + Miss + Overlap + Confusion}{Time} 
\end{equation}

\begin{equation}\label{eq:jer}
    JER_{Speaker} = \frac{False\ Alarm + Miss}{Speech_{Ref} + Speech_{Sys}}
\end{equation}

A Taxa de Erro de Diarização, definida na equação \ref{eq:der}, consiste na fração do tempo total da diarização que não é atribuída corretamente (\textit{False Alarm}), não é identificada (\textit{Miss}), possui sobreposição de locutores (\textit{Overlap}), ou é atribuída ao locutor errado (\textit{Confusion}).
A métrica foi estabelecida pelo \textit{National Institute of Standards and Technology} norte-americano, NIST, e se encontra definida formalmente em \cite{nist2009RT09Rich2009} e \cite{fiscusRichTranscription20062006}, e é desde então a mais utilizada para medição de desempenho em sistemas de diarização.

Já a Taxa de Erro de Jaccard, definida na equação \ref{eq:jer}, consiste no número de segmentos atribuídos incorretamente na diarização produzida em relação à diarização de referência (\textit{False Alarm}), e vice-versa (\textit{Miss}), em relação ao número total de segmentos de fala do mesmo em ambas as diarizações.
Sua definição formal foi feita no artigo introdutório do desafio DIHARD II, e pode ser encontrada em  \cite{ryantSecondDIHARDDiarization2019}.

Para o ajuste do tamanho do passo e algoritmo de consolidação da aplicação, discutidos na seção \ref{sec:class-commit}, assim como a avaliação da aplicação diarizadora como um todo, executamos a mesma sobre um dos vídeos completos para diversas combinações de parâmetros e para cada um dos modelos treinados. 
Além disso, para o cálculo da DER, consideramos um tamanho de passo de $0.033$ segundo, correspondente a um quadro da mídia, e colarinho de $0.1$ segundo.

Os resultados da DER e JER para o vídeo diarizado em função dos parâmetros do diarizador podem observados, respectivamente, nas tabelas \ref{tab:diarization-results-der} e \ref{tab:diarization-results-jer}.
Observe que o melhor algoritmo de consolidação das classificações foi, em todos os casos, o de média das confianças.
Já no tamanho do passo, obtivemos melhores resultados quando este era de 3 quadros para modelos com maior confiança (representados pelo valor mínimo da função custo) ou 1 quadro para o modelo com acurácia máxima.
Por fim, ambos os modelos obtido ao final do treinamento e de valor mínimo da função custo obtiveram o mesmo desempenho quando quantificado pela DER, enquanto o modelo final obteve desempenho ligeiramente melhor (aprox. $1.2\%$) quando avaliado pela JER.

\begin{table}[ht]
    \centering
    \begin{tabular}{|c|c|>{\centering\arraybackslash}m{1.8cm}|>{\centering\arraybackslash}m{1.8cm}|>{\centering\arraybackslash}m{2cm}|>{\centering\arraybackslash}m{1.8cm}|>{\centering\arraybackslash}m{1.8cm}|}
        \hline
        Modelo & Passo & Mediana & Média & Frequência & Média Gauss. & Freq. Gauss. \\
        \hline
        \multirow{4}{*}{\shortstack[c]{Custo\\ Mínimo}} & 1 & 36.08 & 32.80 & 32.90 & 34.78 & 34.65 \\
        \cline{2-7}
        & 3 & 36.31 & {\color{red}\textbf{32.50}} & 32.90 & 35.20 & 34.99 \\
        \cline{2-7}
        & 5 & 37.09 & 35.07 & 35.23 & 37.09 & 37.09 \\
        \cline{2-7}
        & 15 & 40.47 & 40.47 & 40.47 & 40.47 & 40.47\\
        \hline
        \multirow{4}{*}{\shortstack[c]{Acurácia\\ Máxima}} & 1 & 34.86 & {\color{red}\textbf{32.52}} & 32.56 & 33.78 & 33.64 \\
        \cline{2-7}
        & 3 & 35.77 & 34.15 & 34.35 & 34.96 & 34.96 \\
        \cline{2-7}
        & 5 & 36.39 & 36.56 & 36.56 & 36.72 & 36.39 \\
        \cline{2-7}
        & 15 & 39.69 & 39.69 & 39.69 & 39.69 & 39.69 \\
        \hline
        \multirow{4}{*}{Final} & 1 & 35.56 & 32.76 & 32.79 & 33.64 & 33.53 \\
        \cline{2-7}
        & 3 & 35.67 & {\color{red}\textbf{32.50}} & 32.60 & 34.02 & 33.81 \\
        \cline{2-7}
        & 5 & 36.08 & 35.00 & 35.00 & 36.05 & 36.08 \\
        \cline{2-7}
        & 15 & 40.61 & 40.61 & 40.61 & 40.61 & 40.61 \\
        \hline
    \end{tabular}
    \caption{Taxa de Erro de Diarização (DER) para a diarização em função dos parâmetros do diarizador.}
    \label{tab:diarization-results-der}
\end{table}

\begin{table}[ht]
    \centering
    \begin{tabular}{|c|c|>{\centering\arraybackslash}m{1.8cm}|>{\centering\arraybackslash}m{1.8cm}|>{\centering\arraybackslash}m{2cm}|>{\centering\arraybackslash}m{1.8cm}|>{\centering\arraybackslash}m{1.8cm}|}
        \hline
        Modelo & Passo & Mediana & Média & Frequência & Média Gauss. & Freq. Gauss. \\
        \hline
        \multirow{4}{*}{\shortstack[c]{Custo\\ Mínimo}} & 1 & 37.08 & 34.90 & 34.98 & 36.35 & 36.23 \\
        \cline{2-7}
        & 3 & 37.18 & {\color{red}\textbf{34.63}} & 34.97 & 36.51 & 36.36 \\
        \cline{2-7}
        & 5 & 37.91 & 36.78 & 36.91 & 37.91 & 37.91 \\
        \cline{2-7}
        & 15 & 40.90 & 40.90 & 40.90 & 40.90 & 40.90 \\
        \hline
        \multirow{4}{*}{\shortstack[c]{Acurácia\\ Máxima}} & 1 & 36.50 & {\color{red}\textbf{34.93}} & 34.96 & 35.67 & 35.56 \\
        \cline{2-7}
        & 3 & 37.09 & 36.30 & 36.46 & 36.77 & 36.77 \\
        \cline{2-7}
        & 5 & 37.43 & 38.29 & 38.29 & 37.73 & 37.43 \\
        \cline{2-7}
        & 15 & 40.22 & 40.22 & 40.22 & 40.22 & 40.22 \\
        \hline
        \multirow{4}{*}{Final} & 1 & 36.37 & 34.56 & 34.59 & 34.84 & 34.75 \\
        \cline{2-7}
        & 3 & 36.46 & {\color{red}\textbf{34.21}} & 34.29 & 35.25 & 35.09 \\
        \cline{2-7}
        & 5 & 36.58 & 35.99 & 35.99 & 36.49 & 36.58 \\
        \cline{2-7}
        & 15 & 39.46 & 39.46 & 39.46 & 39.46 & 39.46 \\
        \hline
    \end{tabular}
    \caption{Taxa de Erro de Jaccard (JER) para a diarização em função dos parâmetros do diarizador.}
    \label{tab:diarization-results-jer}
\end{table}