\chapter{Introdução}

\section{Descrição do Problema}

A diarização de locutor consiste no processo de identificar os diferentes locutores em um conteúdo multimídia, de forma a separa-los temporalmente, definindo quando quem falou, e produzindo um tipo de roteiro para o mesmo.

Tradicionalmente, tenta-se resolver esse problema através da análise exclusiva do áudio, por meio da extração de \textit{features} na forma de vetores denominados \textit{I-vectors}, e subsequente clusterização destes. Porém, trata-se de um problema difícil; o timbre, principal característica sonora responsável pela identificação do locutor pelo ser humano, é de caráter neurológico \cite{oxenhamPitchPerception2012}, produzido pela decomposição da onda de áudio em seus harmônicos pelo trato auditivo. E, ainda, como propriedade intrínseca da etapa de clusterização, a utilização desses algoritmos depende do conhecimento prévio do número de locutores que participam do áudio. 

Dadas essas limitações, temos que o desempenho dos algoritmos considerados estado da arte é insuficiente, com taxa de erro de diarização (a partir daqui chamada de DER, do inglês \textit{Diarization Error Rate}) de cerca de 20\% \cite{zewoudieUseLongtermFeatures2018}. Portanto, a busca de outras técnicas capazes de prover um melhor desempenho nos leva a considerar também outros sinais constituintes do conteúdo multimídia, como o de vídeo do orador.

\section{Motivação}

Em muitas situações, no processo de transcrição de áudio e vídeo, é interessante obter também a informação de quem está falando. Com essa informação seria possível roteirizar a mídia, viabilizando uma melhor formatação do texto transcrito. Além disso, essa informação adicional permite viabilizar novos critérios de busca sobre o texto transcrito, tornando possível filtrar os resultados por locutor.

\section{Escopo}

De forma geral, gostaríamos de identificar as falas de todos os oradores do texto. No entanto, muitas vezes é suficiente identificar apenas um destes, por exemplo, no caso de audiências e depoimentos do Tribunal de Justiça. Nesse caso especificamente, desejamos identificar com maior precisão os trechos falados pelo depoente, independentemente do número de participantes da audiência, de forma a distinguir futuramente o que foi dito pelo mesmo em seu depoimento.

Para esta finalidade, possuímos um vídeo frontal ou em perfil da face do locutor (depoente), além do áudio combinado de todos os participantes em canal monoaural. Assim, este trabalho ficará limitado a tratar de casos nos quais estas informações estejam disponíveis, e com a determinação de que o vídeo deve ser de boa qualidade.

\section{Estrutura do Trabalho}

Este trabalho se encontra organizado da seguinte maneira:

No capítulo 2 definimos mais a fundo o problema, contextualizando nossa proposta em função das particularidades e limitações de nosso caso. Realizamos também uma revisão bibliográfica, visitando algumas das técnicas atualmente utilizadas para a diarização de locutor, suas vantagens e desvantagens.

No capítulo 3 apresentamos formalmente a nossa proposta, discutindo sua fundamentação teórica e apresentando as técnicas e tecnologias utilizadas em sua implementação. Vemos também o pré-processamento que e feito sobre os dados originais, assim como a arquitetura proposta para a rede de reconhecimento.

No capítulo 4 demonstramos os resultados de nosso método quando aplicado sobre dataset fornecido pela Defensoria Pública do Estado do Rio de Janeiro.

Por fim, no capítulo 5, resumimos e apresentamos nossas conclusões sobre o trabalho realizado, discutindo os resultados obtidos e os fatores que contribuíram para estes, e enumeramos trabalhos futuros a serem realizados sobre o mesmo tema.