%%
%% This is file `example.tex',
%% generated with the docstrip utility.
%%
%% The original source files were:
%%
%% coppe.dtx  (with options: `example')
%% 
%% This is a sample monograph which illustrates the use of `coppe' document
%% class and `coppe-unsrt' BibTeX style.
%% 
%% \CheckSum{1416}
%% \CharacterTable
%%  {Upper-case    \A\B\C\D\E\F\G\H\I\J\K\L\M\N\O\P\Q\R\S\T\U\V\W\X\Y\Z
%%   Lower-case    \a\b\c\d\e\f\g\h\i\j\k\l\m\n\o\p\q\r\s\t\u\v\w\x\y\z
%%   Digits        \0\1\2\3\4\5\6\7\8\9
%%   Exclamation   \!     Double quote  \"     Hash (number) \#
%%   Dollar        \$     Percent       \%     Ampersand     \&
%%   Acute accent  \'     Left paren    \(     Right paren   \)
%%   Asterisk      \*     Plus          \+     Comma         \,
%%   Minus         \-     Point         \.     Solidus       \/
%%   Colon         \:     Semicolon     \;     Less than     \<
%%   Equals        \=     Greater than  \>     Question mark \?
%%   Commercial at \@     Left bracket  \[     Backslash     \\
%%   Right bracket \]     Circumflex    \^     Underscore    \_
%%   Grave accent  \`     Left brace    \{     Vertical bar  \|
%%   Right brace   \}     Tilde         \~}
%%

%% Adapted by Cabral, Felipe G. for undergraduated projects for POLI students

%% Adapted by Lima, Publio M. for undergraduated projects for EQ students

\documentclass[grad,numbers]{coppe}% Check the following table in order to know how to fill the command \documentclass[opt,numbers]{coppe} according to your degree, i.e., undergraduate projects, from POLI or EQ, master or doctor thesis from COPPE.

%% substitute where opt can be:
%% msc     - for Master Dissertations
%% dscexam - for Doctorade Qualification Exams
%% dsc     - for Doctorade Thesis
%% grad    - for Undergraduate Projects for Poli Students
%% gradeq  - for Undergraduate Projects for EQ Students

\usepackage{amsmath,amssymb}
\usepackage{hyperref}
\usepackage[utf8]{inputenc}
\usepackage[brazil]{babel}
\usepackage[T1]{fontenc}
\usepackage{graphicx}
\usepackage{array}
\usepackage{float}
\usepackage[center]{caption}
\usepackage{arydshln}
\usepackage{blkarray}
\usepackage{textcomp}
\graphicspath{{figures/}}
\newtheorem{defi}{Definição}[chapter]
\newtheorem{obse}{Observação}[chapter]
\makelosymbols
\makeloabbreviations

\newcommand{\Inhib}{\textit{Inhib}}

\begin{document}
  \title{Diarização e Identificação de Locutor em Conteúdo de Vídeo Baseada em Análise de Expressão Facial via Aprendizado de Máquina Supervisionado}
  \foreigntitle{Facial Expression Analysis for Speaker Diarization and Identification in Video Content via Supervised Machine Learning}
  \author{Renan Fasolato}{Basilio}
  \advisor{Prof.}{Geraldo Zimbrão}{da Silva}{D.Sc.}

  \examiner{Prof.}{}{}
  \examiner{Prof.}{}{}
  \examiner{Prof.}{}{}
  
  
  \department{ECI}% Confira a tabela a seguir para saber como preencher o comando \department de acordo com seu curso (Graduação - Poli,Graduação - EQ) ou programa (Pós-Graduação - COPPE).
  
  
  %%%%%% Para alunos da EQ %%%%%%
  
  %% Course											Option
  %% Engenharia Química                               EQ
  %% Engenharia de Bioprocessos                       EB
  %% Engenharia de Alimentos                          EAL
  %% Química industrial                               QI
  
  %%%%%% Para alunos da POLI %%%%%%
  
  %% Course											Option
  %% Engenharia Ambiental                             EA
  %% Engenharia Civil                                 ECV
  %% Engenharia de Computação e Informação            ECI
  %% Engenharia de Controle e Automação               ECA
  %% Engenharia de Materiais                          EMAT
  %% Engenharia de Petróleo                           EPT
  %% Engenharia de Produção                           EPR
  %% Engenharia Eletrônica e de Computação            EEC
  %% Engenharia Elétrica                              EET
  %% Engenharia Mecânica                              EMC
  %% Engenharia Metalúrgica                           EMET
  %% Engenharia Naval e Oceânica                      ENO
  %% Engenharia Nuclear                               ENU
  
  
  %%%%%% Para alunos da COPPE %%%%%%
  
  %% Program											Option
  %% Engenharia Biomédica								PEB
  %% Engenharia Civil									PEC
  %% Engenharia Elétrica								PEE
  %% Engenharia Mecânica								PEM
  %% Engenharia Metalúrgica e de Materiais				PEMM
  %% Engenharia Nuclear									PEN
  %% Engenharia Oceânica								PENO
  %% Planejamento Energético							PPE
  %% Engenharia de Produção								PEP
  %% Engenharia Química									PEQ
  %% Engenharia de Sistemas e Computação				PESC
  %% Engenharia de Transportes							PET
  
  
  \date{\the\month}{\the\year}

  \keyword{Aprendizado Supervisionado}
  \keyword{Aprendizado de Máquina}
  \keyword{Diarização de Locutor}

  \maketitle
  
  \frontmatter
  \makecatalog
  \dedication{To-Do: Write Dedication}
  
  \mainmatter

  \chapter{Introdução}

\section{Descrição do Problema}

A diarização de locutor consiste no processo de identificar os diferentes locutores em um conteúdo multimídia, de forma a separa-los temporalmente, definindo quando quem falou, e produzindo um tipo de roteiro para o mesmo.

Tradicionalmente, tenta-se resolver esse problema através da análise exclusiva do áudio, por meio da extração de \textit{features} na forma de vetores denominados \textit{I-vectors}, e subsequente clusterização destes. Porém, trata-se de um problema difícil; o timbre, principal característica sonora responsável pela identificação do locutor pelo ser humano, é de caráter neurológico \cite{oxenhamPitchPerception2012}, produzido pela decomposição da onda de áudio em seus harmônicos pelo trato auditivo. E, ainda, como propriedade intrínseca da etapa de clusterização, a utilização desses algoritmos depende do conhecimento prévio do número de locutores que participam do áudio. 

Dadas essas limitações, temos que o desempenho dos algoritmos considerados estado da arte é insuficiente, com taxa de erro de diarização (a partir daqui chamada de DER, do inglês \textit{Diarization Error Rate}) de cerca de 20\% \cite{zewoudieUseLongtermFeatures2018} nas bases tradicionais. Portanto, a busca de outras técnicas capazes de prover um melhor desempenho nos leva a considerar também outros sinais constituintes do conteúdo multimídia, como o de vídeo do orador.

\section{Motivação}

Em muitas situações, no processo de transcrição de áudio e vídeo, é interessante obter também a informação de quem está falando. Com essa informação seria possível roteirizar a mídia, viabilizando uma melhor formatação do texto transcrito. Além disso, essa informação adicional permite viabilizar novos critérios de busca sobre o texto transcrito, tornando possível filtrar os resultados por locutor.

\section{Escopo}

De forma geral, gostaríamos de identificar as falas de todos os oradores do texto. No entanto, muitas vezes é suficiente identificar apenas um destes, por exemplo, no caso de audiências e depoimentos do Tribunal de Justiça. Nesse caso especificamente, desejamos identificar com maior precisão os trechos falados pelo depoente, independentemente do número de participantes da audiência, de forma a distinguir futuramente o que foi dito pelo mesmo em seu depoimento.

Para esta finalidade, assumimos que esteja disponível um vídeo frontal da face do locutor (depoente), além do áudio combinado de todos os participantes em canal monoaural. Esta suposição é feita tendo em vista as características dos dados reais, sob o conhecimento de que caso o áudio estivesse separado em canais correspondentes a cada locutor o problema se tornaria trivial. Assim, este trabalho ficará limitado a tratar de casos nos quais estas informações estejam disponíveis, e com a determinação de que o vídeo deve ser de boa qualidade.

\section{Estrutura do Trabalho}

Este trabalho se encontra organizado da seguinte maneira:

% TODO: Rework into current chapter layout 

No capítulo 2 definimos mais a fundo o problema, contextualizando nossa proposta em função das particularidades e limitações de nosso caso. Realizamos também uma revisão bibliográfica, visitando algumas das técnicas atualmente utilizadas para a diarização de locutor, suas vantagens e desvantagens.

No capítulo 3 apresentamos formalmente a nossa proposta, discutindo sua fundamentação teórica e apresentando as técnicas e tecnologias utilizadas em sua implementação. Vemos também o pré-processamento que e feito sobre os dados originais, assim como a arquitetura proposta para a rede neural de reconhecimento.

No capítulo 4 demonstramos os resultados de nosso método quando aplicado sobre dataset fornecido pela Defensoria Pública do Estado do Rio de Janeiro, além do dataset AMI Meeting Corpus \cite{mccowanAMIMeetingCorpus2005}, que contém vídeos dos participantes individuais, em alta resolução e com as características desejadas.

Por fim, no capítulo 5, resumimos e apresentamos nossas conclusões sobre o trabalho realizado, discutindo os resultados obtidos e os fatores que contribuíram para estes, e enumeramos trabalhos futuros a serem realizados sobre o mesmo tema.

No apêndice, apresentamos o código fonte desenvolvido para este trabalho.

  \backmatter
  \bibliographystyle{ieeetran}
  \bibliography{citations}

  %\appendix
  %\chapter{Algumas Demonstrações}
  
  
\end{document}
%% 
%%
%% End of file `example.tex'.
