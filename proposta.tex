\chapter{Proposta}
\label{ch:proposal}

\section{Definição do Problema}
\label{sec:problem-desc}

\section{Solução Proposta}
\label{sec:proposed-solution}

% NOTE: em tempo real?
A solução proposta neste trabalho consiste em construir um sistema capaz de diarizar as falas de um único locutor a partir, exclusivamente, de um vídeo frontal do mesmo. Para isso, o sistema de diarização deve ser capaz de, a partir de uma sequência de quadros extraída do vídeo original, determinar se o objeto da gravação está ou não falando. Neste trabalho iremos desconsiderar o áudio associado, já que o processamento deste foge à área de visão computacional.

Com esta finalidade, propomos uma arquitetura em três etapas. 
\begin{enumerate}
    \item Com o uso de um identificador facial, o sistema identificará rostos nas imagens que lhe forem fornecidas. 
    \item Pontos de referência relevantes serão extraídos dos rostos identificados. 
    \item Através de um modelo de aprendizado de máquina previamente treinado, o sistema deverá ser capaz de determinar se o sujeito identificado está ou não falando.
\end{enumerate}

\section{Trabalhos Relacionados}
\label{sec:related-work}
